\section*{Introduction}

KnapSack is a C library providing structures and functions to solve the 0/1 knapsack problem: given a set of item with a cost and value, which subset of items has the maximum total value while the total cost stays under a given budget.\\ 

It uses the \begin{ttfamily}PBErr\end{ttfamily} and \begin{ttfamily}GSet\end{ttfamily} library.\\

\section{Definitions}

The definition of the knapsack problem and its solution can be found on Wikipedia:\\
https://en.wikipedia.org/wiki/Knapsack\_problem

\section{Interface}

\begin{scriptsize}
\begin{ttfamily}
\verbatiminput{/home/bayashi/Coding/KnapSack/knapsack.h}
\end{ttfamily}
\end{scriptsize}

\section{Code}

\subsection{knapsack.c}

\begin{scriptsize}
\begin{ttfamily}
\verbatiminput{/home/bayashi/Coding/KnapSack/knapsack.c}
\end{ttfamily}
\end{scriptsize}

\subsection{knapsack-inline.c}

\begin{scriptsize}
\begin{ttfamily}
\verbatiminput{/home/bayashi/Coding/KnapSack/knapsack-inline.c}
\end{ttfamily}
\end{scriptsize}

\section{Makefile}

\begin{scriptsize}
\begin{ttfamily}
\verbatiminput{/home/bayashi/Coding/KnapSack/Makefile}
\end{ttfamily}
\end{scriptsize}

\section{Unit tests}

\begin{scriptsize}
\begin{ttfamily}
\verbatiminput{/home/bayashi/Coding/KnapSack/main.c}
\end{ttfamily}
\end{scriptsize}

\section{Unit tests output}

\begin{scriptsize}
\begin{ttfamily}
\verbatiminput{/home/bayashi/Coding/KnapSack/unitTestRef.txt}
\end{ttfamily}
\end{scriptsize}
